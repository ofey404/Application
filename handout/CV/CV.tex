%%%%%%%%%%%%%%%%%%%%%%%%%%%%%%%%%%%%%%%%%
% Medium Length Professional CV
% LaTeX Template
% Version 2.0 (8/5/13)
%
% This template has been downloaded from:
% http://www.LaTeXTemplates.com
%
% Original author:
% Trey Hunner (http://www.treyhunner.com/)
%
% Important note:
% This template requires the resume.cls file to be in the same directory as the
% .tex file. The resume.cls file provides the resume style used for structuring the
% document.
%
%%%%%%%%%%%%%%%%%%%%%%%%%%%%%%%%%%%%%%%%%

%----------------------------------------------------------------------------------------
%	PACKAGES AND OTHER DOCUMENT CONFIGURATIONS
%----------------------------------------------------------------------------------------

\documentclass{resume} % Use the custom resume.cls style

\usepackage[left=0.5in,top=0.6in,right=0.5in,bottom=0.6in]{geometry} % Document margins
\newcommand{\tab}[1]{\hspace{.2667\textwidth}\rlap{#1}}
\newcommand{\itab}[1]{\hspace{0em}\rlap{#1}}

% FIXME: bullet is temp solution
\address{ ofey206@gmail.com \bullet +86\ 19921250323 \bullet Fudan\ University,\ Shanghai,\ 200433}
% \address{Fudan University, Shanghai, 200433} % Your address
% \address{ofey206@gmail.com}
% P.S. Or 17307110121@fudan.edu.cn?
\name{Weiwen Chen} % Your name


\begin{document}


%----------------------------------------------------------------------------------------
%	WORK AND RESEARCH EXPERIENCE SECTION
%----------------------------------------------------------------------------------------

\begin{rSection}{Work and Research Experience}


%-------Department of Computer Science---------

\begin{rSubsection}
{Department of Computer Science, Fudan University}{Shanghai, China}
{Member of Data Analysis and Security Lab(Supervisor: Prof. Kai Zhang)} {June 2021-Now}

\item Explored multi tenant on CMU's high-performance Key-Value database mica, with a doctoral student.
\item Introduced the idea of random sampling to speed up load estimation, based on my prior knowledge of Gprof.
\item Upgraded DPDK's incompatible API on mica's 30K line code base, by introducing a test framework.
% \item Made a presentation about image embedding.

\end{rSubsection}

%--------PingCAP Inc.-----------------------

\begin{rSubsection}
% P.S. Should I include a link?
{PingCAP Inc.}{Shanghai, China}
{Intern of Quality Assurance Engineer(Mentor: Shaowen Yin)} {Nov 2020-April 2021}

\item Built a metrics checker for automated test with golang. Used and still maintained after I left.
% https://github.com/PingCAP-QE/metrics-checker

\item Simplified debugging TiDB inside k8s container by building a tool in python, along with my mentor.
% https://github.com/cosven/tidb-testing/tree/master/tipocket-ctl

\item Accomplished end-to-end test with developers in the release of a major version.

% \item Made a presentation about fuzzing.

\end{rSubsection}


%--------Department of Optical Science-----------

\begin{rSubsection}
% P.S. Link of the paper mentioned:
%      https://www.nature.com/articles/s41699-020-0147-x
{Department of Optical Science and Engineering, Fudan University}{Shanghai, China}
{Member of Computational Physics Lab(Supervisor: Prof. Hao Zhang)}
{June 2018-November 2019}

\item Coded for data processing of a paper on npj 2D Materials and Applications, in the name of third author.
% \item Also did miscellaneous, most infrastructural work: 
\item Taught my colleagues workflow of git, set up a private Gitlab. Still invited to give lectures from time to time.
\item Wrote some crawlers and scripts to aggregate data from several researchers' website. Still in use.
% P.S. Maybe too much items here?


\end{rSubsection}


% -------School of Microelectronics--------

\begin{rSubsection}
% SOC: System on chip.
{School of Microelectronics, Fudan University}{Shanghai, China}
{Teaching Assistant of SOC lab(Supervisor: Fei Ye, Camel Microelectronics)}
{March 2018-November 2018}

\item Implemented CAN bus protocol on M2 chip produced by camel microelectronics, with another freshmen. Working with bare address and register, since there is only a C compiler and no operating system.
\item Prepared textbook, recorded video for summer course "SOC system". Sometimes lab members will ask me for help on linux, and newcomers of the lab can always recognize me for the video.
 
\end{rSubsection}



\end{rSection}


%----------------------------------------------------------------------------------------
%	EDUCATION SECTION
%----------------------------------------------------------------------------------------
% P.S. Maybe I should change the order of the education and technical strength section?

\begin{rSection}{Education}

{\bf Fudan University} \hfill {\em August 2017 - Present} 
\\ Bachelor of Physics \hfill { GPA: 3.26/4}
% P.S. Shouldn't I mention my GPA here? If I don't, this section would be empty.
\\ Relevant Courses: Data Structure, Operating System, Computer Network, Introduction to Database.
% P.S. Don't get A. But I don't have other things to put into this section.
% Data Structure and Operating System are Enhancement Courses.

\end{rSection}


%----------------------------------------------------------------------------------------
%	Programming Projects Session
%----------------------------------------------------------------------------------------

\begin{rSection}{Programming Projects }

{\bf PingCAP-QE/metrics-checker}: Check metrics from prometheus, during internship in PingCAP Inc.
\\ {\bf cosven/tidb-testing/tipocket-ctl}: Help developers debug TiDB in k8s container, in PingCAP Inc. 
\\ {\bf ofey404/CFM, ofey404/YAFTP}: Fuzzing tool and subset of FTP protocol, in class coding projects.
\\ {\bf Slack Backup Bot For FDUCSLG}: Tool to backup slack message for CS Lover's Group, Fudan University.
{\em (All hosted on github.)}

\end{rSection}


%----------------------------------------------------------------------------------------
%	TECHNICAL STRENGTHS SECTION
%----------------------------------------------------------------------------------------

\begin{rSection}{Skills}

{\bf Programming Languages(Ordered by proficiency):} python, cpp, golang, shell, rust, javascript and lisp.
\\ {\bf Knowledge:} Database, operating system, algorithm, data structure, UNIX concepts, low level programming tools and other common sense that help me survive in the computer world.
\\ {\bf Language:} Chinese, English.
\\ {\bf I learn things fast.}
\\ {\bf Extracurriculars:} Vim, Emacs, VSCode, Gnome Desktop, guitar, bass and bypassing GFW. Picking up drawing now.

\end{rSection}
\end{document}
