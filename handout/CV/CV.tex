%%%%%%%%%%%%%%%%%%%%%%%%%%%%%%%%%%%%%%%%%
% Medium Length Professional CV
% LaTeX Template
% Version 2.0 (8/5/13)
%
% This template has been downloaded from:
% http://www.LaTeXTemplates.com
%
% Original author:
% Trey Hunner (http://www.treyhunner.com/)
%
% Important note:
% This template requires the resume.cls file to be in the same directory as the
% .tex file. The resume.cls file provides the resume style used for structuring the
% document.
%
%%%%%%%%%%%%%%%%%%%%%%%%%%%%%%%%%%%%%%%%%

%----------------------------------------------------------------------------------------
%	PACKAGES AND OTHER DOCUMENT CONFIGURATIONS
%----------------------------------------------------------------------------------------

\documentclass{resume} % Use the custom resume.cls style

\usepackage[left=0.5in,top=0.6in,right=0.5in,bottom=0.6in]{geometry} % Document margins
\newcommand{\tab}[1]{\hspace{.2667\textwidth}\rlap{#1}}
\newcommand{\itab}[1]{\hspace{0em}\rlap{#1}}

\address{ +86 19921250323 }
\address{Fudan University, Shanghai, 200433} % Your address
\address{ofey206@gmail.com}
% P.S. Or 17307110121@fudan.edu.cn?
\name{Weiwen Chen} % Your name


\begin{document}

%----------------------------------------------------------------------------------------
%	EDUCATION SECTION
%----------------------------------------------------------------------------------------
% P.S. Maybe I should change the order of the education and technical strength section?
\begin{rSection}{Education}

{\bf Fudan University} \hfill {\em August 2017 - Present} 
\\ Bachelor of Physics \hfill { GPA: 3.26/4}
% P.S. Shouldn't I mention my GPA here? If I don't, this section would be empty.
\\ Relevant Courses: Data Structure, Operating System, Computer Network, Introduction to Database.
% P.S. Don't get A. But I don't have other things to put into this section.
% Data Structure and Operating System are Enhancement Courses.

\end{rSection}


%----------------------------------------------------------------------------------------
%	WORK AND RESEARCH EXPERIENCE SECTION
%----------------------------------------------------------------------------------------

\begin{rSection}{Work and Research Experience}

\begin{rSubsection}
% FIXME: I cannot find Prof. Ye on google. Maybe I should find another teacher in lab as my supervisor?
%        Only webpage I found relate to him is that:
%        https://www.matec-conferences.org/articles/matecconf/abs/2018/91/matecconf_eitce2018_04079/matecconf_eitce2018_04079.html
% SOC: System on chip.
{School of Microelectronics, Fudan University}{Shanghai, China}
{Teaching Assistant of SOC lab(Supervisor: Fei Ye, Camel Microelectronics)}
{March 2018-November 2018}

\item Implemented CAN bus protocol on M2 chip produced by camel microelectronics, with another freshmen. Working with bare address and register, since there is only a C compiler and no operating system.
\item Prepared textbook, recorded video for summer course "SOC system". Those materials are still in use.
 
\end{rSubsection}


%------------------------------------------------

\begin{rSubsection}
% P.S. Link of the paper mentioned:
%      https://www.nature.com/articles/s41699-020-0147-x
{Department of Optical Science and Engineering, Fudan University}{Shanghai, China}
{Member of Computational Physics Lab(Supervisor: Prof. Hao Zhang)}
{June 2018-November 2019}

\item Coded for data processing of a paper on npj 2D Materials and Applications, in the name of third author.
% \item Also did miscellaneous, most infrastructural work: 
\item Set up a private gitlab on the server of lab, wrote introduction documentation, and gave lectures about git.
\item Wrote some crawlers and scripts to aggregate data from several researchers' website.
% P.S. Maybe too much items here?


\end{rSubsection}

%------------------------------------------------

\begin{rSubsection}
% P.S. Should I include a link?
{PingCAP Inc.}{Shanghai, China}
{Intern of Quality Assurance Engineer(Mentor: Shaowen Yin)} {Nov 2020-April 2021}

\item Built a metrics checker for automated test with golang.
% https://github.com/PingCAP-QE/metrics-checker

\item Built a tool, along with my mentor, to help developers debug inside the k8s container, with python and shell.
% https://github.com/cosven/tidb-testing/tree/master/tipocket-ctl

\item Did end-to-end test with developers in the release of a major version.

% \item Made a presentation about fuzzing.

\end{rSubsection}


\begin{rSubsection}
{Department of Computer Science, Fudan University}{Shanghai, China}
{Member of Data Analysis and Security Lab(Supervisor: Prof. Kai Zhang)} {June 2021-Now}

\item Tried to share the CMU's high-performance Key-Value database mica by multi tenants, with a doctoral student.
\item Upgraded mica's DPDK support by introducing a test framework. It stuck the progress for about one month.
% \item Made a presentation about image embedding.

\end{rSubsection}

\end{rSection}

%----------------------------------------------------------------------------------------
%	Programming Projects Session
%----------------------------------------------------------------------------------------

\begin{rSection}{Programming Projects}

{\bf PingCAP-QE/metrics-checker}: Check metrics from prometheus, during internship in PingCAP Inc.
\\ {\bf cosven/tidb-testing/tipocket-ctl}: Tool to help developers debug TiDB in k8s container, during internship in PingCAP Inc. 
\\ {\bf ofey404/CFM}: Simple fuzzing tool in rust, coding project of Introduction of Computer System Course.
\\ {\bf ofey404/YAFTP}: A subset of FTP protocol in python, coding project of Computer Network Course.
\\ {\bf Slack Backup Bot For FDUCSLG}: Tool to backup slack message for CS Lover's Group, Fudan University.

\end{rSection}


%----------------------------------------------------------------------------------------
%	TECHNICAL STRENGTHS SECTION
%----------------------------------------------------------------------------------------

\begin{rSection}{Skills}

{\bf Programming Languages(Ordered by proficiency):} python, cpp, golang, shell, rust, javascript and lisp.
\\ {\bf Knowledge:} Database, operating system, algorithm, data structure, UNIX concepts, low level programming tools and other common sense that help me survive in the computer world.
\\ {\bf Language:} Chinese, English. Want to learn some Japanese.
\\
\\ {\bf I learn things fast.}
\\
\\ {\bf Extracurriculars:} Vim, Emacs, VSCode, Gnome Desktop, guitar, bass and Bypassing GFW. Picking up drawing now, just follow the step of Feynman.

\end{rSection}


%----------------------------------------------------------------------------------------
\begin{rSection}{Miscellanous} 
I love manga, want to build a version control system for CG project files, "a git for manga artists".

This project, if it's really be made, is dedicated to Miura Kentaro, author of the great manga Berserk. He died early at May 2021, a cooperation in fan-fiction community may be the only way to finish his posthumous work.

\end{rSection}

\end{document}
