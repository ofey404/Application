%%%%%%%%%%%%%%%%%%%%%%%%%%%%%%%%%%%%%%%%%
% Medium Length Professional CV
% LaTeX Template
% Version 2.0 (8/5/13)
%
% This template has been downloaded from:
% http://www.LaTeXTemplates.com
%
% Original author:
% Trey Hunner (http://www.treyhunner.com/)
%
% Important note:
% This template requires the resume.cls file to be in the same directory as the
% .tex file. The resume.cls file provides the resume style used for structuring the
% document.
%
%%%%%%%%%%%%%%%%%%%%%%%%%%%%%%%%%%%%%%%%%

%----------------------------------------------------------------------------------------
%	PACKAGES AND OTHER DOCUMENT CONFIGURATIONS
%----------------------------------------------------------------------------------------

\documentclass{resume} % Use the custom resume.cls style

\usepackage[left=0.5in,top=0.6in,right=0.5in,bottom=0.6in]{geometry} % Document margins
\newcommand{\tab}[1]{\hspace{.2667\textwidth}\rlap{#1}}
\newcommand{\itab}[1]{\hspace{0em}\rlap{#1}}

% FIXME: bullet is temp solution
\address{ ofey206@gmail.com \bullet +86\ 199\ 2125\ 0323}
% \address{Fudan University, Shanghai, 200433} % Your address
% \address{ofey206@gmail.com}
% P.S. Or 17307110121@fudan.edu.cn?
\name{Weiwen Chen} % Your name


\begin{document}

%----------------------------------------------------------------------------------------
%	EDUCATION SECTION
%----------------------------------------------------------------------------------------
% P.S. Maybe I should change the order of the education and technical strength section?

\begin{rSection}{Education}

{\bf Fudan University} \hfill {\em Aug 2017\ -\ Now} 
\\ Bachelor of Physics
% P.S. Shouldn't I mention my GPA here? If I don't, this section would be empty.
\\ Relevant Courses: Data Structure, Operating System, Computer Network, Introduction to Database.
% P.S. Don't get A. But I don't have other things to put into this section.
% Data Structure and Operating System are Enhancement Courses.

\end{rSection}



%----------------------------------------------------------------------------------------
%	WORK AND RESEARCH EXPERIENCE SECTION
%----------------------------------------------------------------------------------------

\begin{rSection}{Work and Research Experience}


%-------Department of Computer Science---------

\begin{rSubsection}
{Department of Computer Science, Fudan University}{Shanghai, China}
{Member of Data Analysis and Security Lab (Supervisor: Prof. Kai Zhang)} {Jun 2021\ -\ Now}

\item Developing multi tenant functionality on CMU's high-performance Key-Value database mica, with a doctoral student.
\item Implemented a request sampling strategy to estimate load effectively in mica, inspired by Gprof.
\item Upgraded DPDK's incompatible API on mica's 30K line code base, by introducing a test framework.

\end{rSubsection}


%--------PingCAP Inc.-----------------------

\begin{rSubsection}
% P.S. Should I include a link?
{PingCAP Inc.}{Shanghai, China}
{Intern of Quality Assurance Engineer (Mentor: Shaowen Yin)} {Nov 2020\ -\ Apr 2021
\\  Projects: github.com/PingCAP-QE/metrics-checker github.com/cosven/tidb-testing/tipocket-ctl
}

\item Built a metrics checker for TiDB's automated test. Still maintained and used in pipeline after I left.
\item Simplified debugging TiDB inside k8s container by developing a tool along with my mentor.
\item Accomplished end-to-end test with developers in the release of a major version on time.
% \item Return offer.


% \item Made a presentation about fuzzing.

\end{rSubsection}


%--------Department of Optical Science-----------

\begin{rSubsection}
% P.S. Link of the paper mentioned:
%      https://www.nature.com/articles/s41699-020-0147-x
{Department of Optical Science and Engineering, Fudan University}{Shanghai, China}
{Member of Computational Physics Lab (Supervisor: Prof. Hao Zhang)}
{Jun 2018\ -\ Nov 2019
\\ Publication: www.nature.com/articles/s41699-020-0147-x }

\item Coded for data processing of a paper on computational material research in the name of third author.
\item Trained all colleagues workflow of git and set up a private Gitlab. Still invited to give lectures about CS.
\item Created some crawlers and scripts to aggregate data from several researchers' website. Still in use.


\end{rSubsection}


% -------School of Microelectronics--------

\begin{rSubsection}
% SOC: System on chip.
{School of Microelectronics, Fudan University}{Shanghai, China}
{Teaching Assistant of SOC lab (Supervisor: Fei Ye, Camel Microelectronics)}
{Mar 2018\ -\ Nov 2018}

\item Implemented CAN bus protocol on Camel's M2 chip, which has operating system, only a C compiler.

\item Organized teaching material into a textbook and demonstrated them by videos, as training material for newcomers.
 
\end{rSubsection}

{\bf Department of Physics, Fudan University } \hfill {\em Sept 2017\ -\ Now}
\begin{myItemize}
\item Built an atmosphere simulation and visualization, in python, at Statistical Physics lesson.
\item Wrote an fuzzing tool, with rust,  on Computer Architecture lesson.
\item Implemented a subset of FTP protocol, in python, as Project of Computer Network lesson.
\end{myItemize}
\end{rSection}

%----------------------------------------------------------------------------------------
%	MISC STRENGTHS SECTION
%----------------------------------------------------------------------------------------

\begin{rSection}{Extracurriculum Experience}

{\bf PC Service of Fudan University } \hfill {\em Mar 2017\ -\ June 2019}

  \begin{list}{$\cdot$}{\leftmargin=0em} % \cdot used for bullets, no indentation
   \itemsep -0.5em \vspace{-0.5em} % Compress items in list together for aesthetics
\item Volunteered giving consultancy on python and system management.

  \end{list}
  
\\ {\bf Physics Experiment Teaching Center } \hfill {\em July 2019}

  \begin{list}{$\cdot$}{\leftmargin=0em} % \cdot used for bullets, no indentation
   \itemsep -0.5em \vspace{-0.5em} % Compress items in list together for aesthetics
\item Automated oil and water droplet classification under microscope with machine learning.
  \end{list}
  
% 和物理实验教学中心的老师关系很好,和他们合作过一个改进显微镜的项目——用机器学习分辨气泡和油滴。
% 没有做过班干部/学生工作。
% 不是党员、团员。
% 会一些吉他/贝司。偶尔会去听学校的摇滚音乐节,参加过乐手联盟(社团)组织的乐理和合奏课程。
% 按篇幅备选。和舆图社(一个地理社团)去过长株潭地区调研。
% 和两任天文社长是好朋友,一起出去看星星。
% 


\end{rSection}



%----------------------------------------------------------------------------------------
%	TECHNICAL STRENGTHS SECTION
%----------------------------------------------------------------------------------------

\begin{rSection}{Skills}

{\bf Programming Languages (in proficiency order):} python, cpp, golang, shell, rust, javascript and lisp.
\\ {\bf Knowledge:} Database, operating system, algorithm, data structure.
\\ {\bf Language:} Chinese (Native), English (TOEFL 105).
\\ {\bf Extracurriculars:} Vim, Emacs, VSCode, Gnome Desktop, guitar, bass and bypassing GFW.

\end{rSection}



\end{document}OC lab